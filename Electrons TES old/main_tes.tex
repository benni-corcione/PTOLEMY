%\documentclass[12p]{article}

\documentclass[aps,prl,reprint,superscriptaddress]{revtex4-1}
\usepackage{CJK}


\usepackage{amsmath}
\usepackage{amssymb}
\usepackage{graphicx}


\begin{document}
\begin{CJK*}{GB}{} % Use default fonts from CJK (see below)


\title{First Detection of Low-Energy Electrons with Transition Edge Sensors}


\author{Carlo Pepe}
%\email{c.pepe@inrim.it}
\affiliation{Istituto Nazionale di Ricerca Metrologica, Strada delle Cacce 91, 10135 Torino}

\author{Benedetta Corcione}
\affiliation{Sapienza Universit\`{a} di Roma, Piazzale Aldo Moro 2, 00185 Rome, Italy}
\affiliation{Istituto Nazionale di Fisica Nucleare - Sezione di Roma, Piazzale Aldo Moro 2, 00185 Rome, Italy}

\author{Francesco Pandolfi}
\email{francesco.pandolfi@roma1.infn.it}
\affiliation{Istituto Nazionale di Fisica Nucleare - Sezione di Roma, Piazzale Aldo Moro 2, 00185 Rome, Italy}


\author{Hobey Garrone}
\affiliation{Istituto Nazionale di Ricerca Metrologica, Strada delle Cacce 91, 10135 Torino}

\author{Eugenio Monticone}
\affiliation{Istituto Nazionale di Ricerca Metrologica, Strada delle Cacce 91, 10135 Torino}

\author{Ilaria Rago}
\affiliation{Istituto Nazionale di Fisica Nucleare - Sezione di Roma, Piazzale Aldo Moro 2, 00185 Rome, Italy}

\author{Gianluca Cavoto}
\affiliation{Sapienza Universit\`{a} di Roma, Piazzale Aldo Moro 2, 00185 Rome, Italy}
\affiliation{Istituto Nazionale di Fisica Nucleare - Sezione di Roma, Piazzale Aldo Moro 2, 00185 Rome, Italy}

\author{Alice Apponi}
\affiliation{Dipartimento di Scienze Universit\`a degli Studi Roma Tre, and Istituto Nazionale di Fisica Nucleare - Sezione di Roma Tre, Via della Vasca Navale 84, 00146 Rome, Italy}


\author{Mauro Rajteri}
\affiliation{Istituto Nazionale di Ricerca Metrologica, Strada delle Cacce 91, 10135 Torino}






\begin{abstract} %PRL max 600 characters, nature max 200 words

We present the first detection of electrons with kinetic energy in the 100~eV range with transition edge sensors (TES).  This has been achieved with a $100\times 100$~$\mu$m$^2$ Ti-Au bilayer TES, with a critical temperature of about 83~mK. The electrons are produced by a cold cathode source based on field emission from vertically-alighed multiwall carbon nanotubes. The obtained energy resolution is smaller than 2~eV for fully-contained electrons, while significant non-containment is observed. This measurement opens new possibilities in the field of electron detection, and constitutes the first detection of such particles without any form of particle multiplication.

\end{abstract}

\maketitle


%main article: max 3,750 words for PRL, max 6 pages for nature

Transition edge sensors (TES) are highly sensitive micro-calorimeters capable of high-resolution single-photon counting across a wide energy spectrum. The detection scheme is based on the absorption of the incoming photons in a thin superconducting film, in which their energy is transformed into heat. By operating a TES device at its critical temperature $T_C$, even small variations in temperature lead to measurable changes in its electrical resistance, owing to the steep transition between the superconducting regime and the normal-conduction one. TES devices have been capable of achieving single-photon energy resolutions, defined throughout this work as the standard deviation of the distribution, as low as $\sigma = 0.X$~eV \cite{}. In principle, this detection scheme should also be sensitive to electrons absorbed in the superconducting film, yet there is very little research on TES devices used in electron detection. The use of TES devices in electron detection could be of great interest to a vast number of experiments, such as, for example, the PTOLEMY collaboration \cite{}, which plans to search for the cosmic neutrino background by analyzing the endpoint of the beta decay of tritium with unprecedented electron energy resolution.


\begin{figure}[tb]
\includegraphics[width=0.5\textwidth]{figures/Electron counting setup.pdf}
\caption{Top: schematic view of the TES device and its shielding layer (image is not to scale). Bottom: schematic view of the setup use for electron counting: the carbon nanotubes (CNTs) are hosted on the top copper plate and oriented with the tips pointing towards the TES.\label{fig:setup}}
\end{figure}


The results presented in this work were obtained in the innovative cryogenic detectors laboratory of the photometry division of Istituto Nazionale di Ricerca Metrologica (INRiM) in Torino. The TES detectors operate on the 30~mK stage of a commercial pulse-tube driven adiabatic demagnetization refrigerator cryostat (Model 103-RC Rainier from HPD \cite{}). The TES device used in this work has a size of $100\times 100$~$\mu$m$^2$, and is a Ti-Au bilayer device, composed by 15~nm of titanium covered by 30~nm of gold, fabricated at INRiM by thermal evaporation on a 500~nm silicon nitride substrate \cite{}. It is wired with 50~nm superconducting niobium (Nb) traces, which were deposited on the substrate via sputtering. The TES has a critical temperature $T_C = 84$~mK and was calibrated with 406~nm photons, obtaining an energy resolution between 0.7 and 1.2~eV in the $0-100$~eV energy range.

This design was adapted for electron detection by adding a shield layer, which is needed as the electron source has a significantly larger area (approximately $1\times 1$~mm$^2$) compared to the TES active area ($100\times 100$~$\mu$m$^2$). Therefore it is necessary to avoid direct electron hits on the Nb wiring, which would induce electric noise, and also to cover the surface of the substrate surrounding the TES, which is insulating at cryogenic temperatures and therefore exposing it to a direct electron current would lead to charge build-up. The shield layer is produced by first depositing an insulating layer consisting of 300~nm of silicon oxide (SiO$_\mathrm{x}$), followed by a thin (5~nm) layer of titanium, and finally a 50~nm layer of gold. The thin titanium layer is necessary for best adhesion to the SiO$_\mathrm{x}$. A schematic view of the detector and the shield layer is shown in the top panel of Figure~\ref{fig:setup}. 
%in order to avoid charge accumulationtherefore consisting of an insulating XX layer, covered by an additional Ti-Au conducting bilayer. The shield leaves exposed the sensitive area of the TES while protecting the wiring from direct electron hits, which would induce additional electrical noise, and also preventing charge accumulation on the insulating substrate.

The electron source consists of a sample of vertically-aligned multi-wall carbon nanotubes which were synthesized in the INFN laboratory `TITAN' at Sapienza University of Rome. The nanotubes were grown through chemical vapor deposition on a 500~$\mu$m silicon substrate, and are approximately 200~microns in length, while covering a surface of roughly $1\times 1$~mm$^2$. Thanks to the high geometrical field enhancement factor of their tips, nanotubes are capable of emitting electrons through quantum tunneling (field emission) without the necessity of applying very high voltages. Furthermore, as this emission is quantic in nature, and not thermal, it does not generate heat and can therefore be used in a cryostat.

The TES and the nanotubes were placed on two copper plates, facing each other, separated by 0.5~mm sapphire spacers, which ensure electrical insulation while guaranteeing a good degree of thermal conductance. The top copper plate, where the nanotubes are hosted, was connected to the power supply of a Keithley 6487, and was provided negative voltage $V_{\mathrm{cnt}}$ in order to produce field-emission electrons; the bottom plate  was put in thermal contact with the cryostat, and grounded electrically through it. The distance between the tips of the nanotubes and the surface of the TES, in this setup, is $d = 0.5$~mm. A schematic view of the setup is shown in the bottom panel of Figure~\ref{fig:setup}.

The TES was read-out in two different ways. In the `anode' configuration it is short-circuited to the metallic layer of the shield layer, and the two are used as a large-area metallic plate which is in turn connected to the Keithley 6487 picoammeter to measure the current $I_{\mathrm{cnt}}$ emitted by the nanotubes. In the `counting' configuration the TES is sensitive to single-particle signals. This is achieved by bringing the TES to a temperature equal to $T_C$ by circulating a current $I_{\mathrm{tes}}$ through it until it reaches its nominal working point, defined as the TES having a resistance $R_{\mathrm{tes}} = 0.35\cdot R_N$, where $R_N$ is the resistance of the TES in normal mode. The TES is then read out with a DC-SQUID transimpedence amplifier. In counting mode the metallic shield is grounded through the cryostat.

\begin{figure}[tb]
\includegraphics[width=0.4\textwidth]{figures/rate.pdf}
\caption{Current $I_{\mathrm{cnt}}$ emitted by the nanotube source (black curve, right vertical scale), as a function of the negative voltage $V_{\mathrm{cnt}}$ provided to it, when reading the TES and the shield in  anode configuration, compared to the rate of signals $\mathrm{d}S/\mathrm{d}t$ (cyan markers, left vertical scale) recorded by the TES read-out in counting configuration. \label{fig:rate}}
\end{figure}

The results in the anode configuration are summarized in the black curve of Figure~\ref{fig:rate}, where $I_{\mathrm{cnt}}$ is shown as a function of $V_{\mathrm{cnt}}$. As can be seen, $I_{\mathrm{cnt}}$ exhibits an exponential rise, compatibly with the Fowler-Nordheim theory on field emission~\cite{}. Superimposed on the same plot with cyan markers is the rate of signals $\mathrm{d}S/\mathrm{d}t$ measured with the TES in counting mode. As can be seen the increase in $\mathrm{d}S/\mathrm{d}t$ as a function of $V_{\mathrm{cnt}}$ follows the same exponential rise as that of $I_{\mathrm{cnt}}$, therefore proving that the signals recorded by the TES are due to electrons.

\begin{figure}[tb]
\includegraphics[width=0.4\textwidth]{figures/temp_power}
\caption{Power needed to bring the TES to its critical temperature (black markers, left vertical scale) and local temperature of the substrate under the TES device (red markers, right vertical scale) for different values of $V_{\mathrm{cnt}}$. The bath temperature is marked with a dashed line. \label{fig:heat}}
\end{figure}

A feature of Fowler-Nordheim emission is that the current of emitted electrons by the nanotubes depends on the electric field $|\vec{E}|$ in proximity of their tips, which in our planar configuration corresponds, in first approximation, to $|\vec{E}| = V_{\mathrm{cnt}}/d$.
%between $V_{\mathrm{cnt}}$ and $d$, which are, respectively, the potential difference and the distance between the nanotube tips and the TES surface.
At the same time, if we assume that electrons are emitted by the nanotubes with a null initial kinetic energy, $V_{\mathrm{cnt}}$ also determines the kinetic energy of the electrons when entering the TES. Therefore in this setup the signal rate and electron energy are not independent parameters, as they both depend on $V_{\mathrm{cnt}}$. 

As the electrons are emitted by a relatively large area compared to the TES, when $V_{\mathrm{cnt}}$ is raised the high rate of electrons hitting the TES and its surroundings produces heat. This means that to bring the TES to its nominal working point a smaller $I_{\mathrm{tes}}$ is needed, as its temperature is higher to begin with. This is shown in Figure~\ref{fig:heat}, where the black markers, which represent the power needed to bring the TES to its nominal working point, can be seen clearly decreasing as $V_{\mathrm{cnt}}$ increases. This dissipated power can in turn be interpreted as the local temperature of the substrate in direct contact with the TES, which is graphed with blue markers in Figure~\ref{fig:heat}: as can be seen, when operating the electron source at $V_{\mathrm{cnt}} = 105$~V the local temperature of the substrate is already higher than 60~mK, compared to the initial temperature of about 41~mK before electron emission. This implies that results at different $V_{\mathrm{cnt}}$ are not rigorously comparable, as the TES is operating in slightly different conditions.

%will heat up the surrounding, in turn, can be interpreted as a change in the temperature of the substrate surrounding the TES As the TES energy resolution depends on the inverse of the current flowing through it \cite{}, which alters the working point of the TES. This can be seen in Figure~\ref{fig:heat}, where the TES operating curves are shown in the plane defined by the current flowing through the TES $I_{\mathrm{tes}}$ and the bias current $I_{bias}$ provided to the circuit, for different values of $V_{\mathrm{cnt}}$. In order to compensate this effect, the TES operating point was changed for every value of $V_{\mathrm{cnt}}$, in such a way that its resistance $R_{\mathrm{tes}}$ was equal to 35\% of its normal resistance $R_N$ (shown as a red line in  Figure~\ref{fig:heat}).

\begin{figure}[tb]
\includegraphics[width=0.45\textwidth]{figures/spectrum.pdf}
\caption{Typical spectrum of TES signal amplitudes. This spectrum was obtained with $V_{\mathrm{cnt}} = 100$~V.\label{fig:spectrum}}
\end{figure}


For each value of $V_{\mathrm{cnt}}$, the amplitude of the signals produced by the TES were analyzed. A typical spectrum, obtained for $V_{\mathrm{cnt}} = 100$~V is shown in Figure~\ref{fig:spectrum}: as can be seen the distribution presents a high-amplitude peak, corresponding to the full absorption of the electron energy in the sensitive layer of the TES; a marked tail to the left of the peak, due to partial absorption of electrons, and is most likely due to longitudinal non-containment, as the TES has a very thin gold layer (30~nm); and a low-amplitude peak, which is truncated by the trigger threshold of 166~mV.

\begin{figure}[tb]
\includegraphics[width=0.4\textwidth]{figures/fits}
\caption{Example fits, for four different values of $V_{\mathrm{cnt}}$, of the high-amplitude peak with the asymmetric Gaussian function.\label{fig:fits}}
\end{figure}

We fit the distributions with an asymmetric Gaussian function, described by its peak position $\mu$ and its left ($\sigma_L$) and right ($\sigma_R$) tails. Some example fits are shown in Figure~\ref{fig:fits}. 

The trend of the fitted value of $\mu$ is shown in the left panel of Figure~\ref{fig:mu} shows the fitted values of $\mu$, as a function of the nominal energy $E_e$ of the electrons, taken as $E_e = V_{\mathrm{cnt}} \cdot \frac{C}{e}$, where $C$ is the Coulomb charge and $e$ is the charge of the electron. When operating the TES in the same conditions, the position of the absorption peak $\mu$ should increase linearly with the energy of the absorbed electrons, therefore $\mu \propto V_{\mathrm{cnt}}$. However, as explained previously, the working conditions of the TES were not exactly the same for each value of $V_{\mathrm{cnt}}$, therefore some deviations from the linear trend are expected. Nevertheless, we observe a significant rise of the absorption peak position $\mu$. 

%We fit the absorption peak with a Cruijff function:
%$$
%  f(x)=\left\{
%                \begin{array}{ll}
%                  A\cdot \exp \left(-\frac{(x-\mu)^2}{2\sigma_L^2 + \alpha\cdot(x-\mu)^2}\right) \quad \mathrm{for} \,\,x<\mu\\
%                  A\cdot \exp \left(-\frac{(x-\mu)^2}{2\sigma_R^2}\right) \quad \mathrm{for} \,\,x>\mu .\\
%                \end{array}
%              \right.
%$$
%


\begin{figure}[b]
\includegraphics[width=0.23\textwidth]{figures/mu}
\includegraphics[width=0.23\textwidth]{figures/sigma}
\caption{Fitted position of the absorption peak $\mu$ (left) and TES electron energy resolution $\sigma_e$ (right) as a function of the nominal electron energy $E_e$.\label{fig:mu}}
\end{figure}


%\begin{figure}[tb]
%\includegraphics[width=0.45\textwidth]{figures/sigma}
%\caption{Energy resolution, defined as $\sigma = (\sigma_R/\mu)\cdot E_e$, as a function of the  nominal electron energy $E_e$.\label{fig:sigma}}
%\end{figure}

Finally, the left panel of Figure~\ref{fig:mu} shows the trend of $\sigma_e = (\sigma_R/\mu)\cdot E_e$ as a function of $E_e$. The parameter $\sigma_R$ describes the broadness of the high-energy tail of the absorption peak in the amplitude distribution. While the left tail is dominated by electron non-containment effects, the right tail is dominated by the energy resolution of the device, plus possible small effects due to the non-monochromaticity of the source. When rescaling $\sigma_R$ to the nominal energy $E_e$ we obtain $0.8 < \sigma_e < 1.8$~eV for all values of $E_e$, and this can be interpreted as an estimate of the absolute energy resolution of the TES device when detecting electrons in the 100~eV energy range.

AGGIUNGERE LE CONCLUSIONI


\begin{thebibliography}{99}

%\bibitem{deHeer1995} W. A. de Heer, A. Chatelain, D. Ugarte, A Carbon Nanotube Field-Emission Electron Source. Science 1995, 270, 1179-1180
%
%\bibitem{dresselhaus2004}  M.S. Dresselhaus, G. Dresselhaus, J.C. Charlier, E. Hernandez, Electronic, thermal and mechanical properties of carbon nanotubes. Philosophical Transactions of the Royal Society of London, Series A: Mathematical, Physical and Engineering Sciences 362, 2065 (2004)
%
%\bibitem{lei2015} W. Lei, Z. Zhu, C. Liu, X. Zhang, B. Wang, A. Nathan, High current field-emission of carbon nanotubes and its application as a fast-imaging X-ray source. Carbon 94, 687 (2015)
%
%\bibitem{swanson2005} L.W. Swanson, G.A. Schwind, A Review of Field Electron Source Use in Electron Microscopes.
%Microsc. Microanal. 2005, 11
%
%\bibitem{neupane2012} S. Neupane, M. Lastres, M. Chiarella, W. Li, Q. Su, G. Du, Synthesis and field emission properties of vertically aligned carbon nanotube arrays on copper, Carbon 5, 2641-2650, 2012
%
%\bibitem{lee2001} Lee NS, Chung DS, Han IT, Kang JH, Choi YS, Kim JM, et al. Application of carbon nanotubes to field emission displays. Diamond Relat Mater 2001;10(2):265-70
%
%\bibitem{gupta2018} B. K. Gupta, G. Kedawat, A. K. Gangwar, K. Nagpal, P. K. Kashyap, S. Srivastava, S. Singh, P. Kumar, S. R. Suryawanshi, D. M. Seo, P. Tripathi, M. A. More, O. N. Srivastava, M.  G. Hahm,  D. J. Late. (2018). High-performance field emission device utilizing vertically aligned carbon nanotubes-based pillar architectures. AIP Advances. 8. 015117. 10.1063/1.5004769
%
%\bibitem{chen2013} G. Chen, S. Neupane, W. Li, L. Chen, J. Zhang, An increase in the field emission from vertically aligned multiwalled carbon nanotubes caused by NH$_3$ plasma treatment, Carbon, 52, 468-475, 2013
%
%\bibitem{giubileo2018} F. Giubileo, A. Di Bartolomeo, L. Iemmo, G. Luongo, F. Urban, Field Emission from Carbon Nanostructures, Appl. Sci. 2018, 8, 526; doi:10.3390/app8040526
%
%\bibitem{sridhar2014} S. Sridhar, C. Tiwary, S. Vinod, J.J. Taha-Tijerina, S. Sridhar, K. Kalaga, B. Sirota, A.H. Hart, S. Ozden, R.K. Sinha, R. Vajtai, Field emission with ultralow turn on voltage from metal decorated carbon nanotubes, ACS Nano 8, 7763 (2014)
%
%\bibitem{gautier2015} L.A. Gautier, V. Le Borgne, N. Delegan, R. Pandiyan, M.A. El, Khakani, Field electron emission enhancement of graphenated MWCNTs emitters following their decoration with Au nanoparticles by a pulsed laser ablation process, Nanotechnology 26, 045706 (2015)
%
%\bibitem{zeng2006} Zeng BQ, Xiong GY, Chen S, Wang WZ, Wang DZ, Ren ZF. Enhancement of field emission of aligned carbon nanotubes by thermal oxidation. Appl Phys Lett 2006;89(22):223119.
%
%\bibitem{wu2019} X. Wu, H. Yin, Q. Li, Ablation and Patterning of Carbon Nanotube Film by Femtosecond Laser Irradiation, Appl. Sci. 2019, 9, 3045; doi:10.3390/app9153045
%
%\bibitem{yi2019} C. Yi, H. Wu, J. Li, Y. Song, Y. Song, X. Chen, W. Ou-Yang, Crack-Assisted Field Emission Enhancement of Carbon Nanotube Films for Vacuum Electronics, ACS Appl. Nano Mater. 2019, 2, 7803-7809
%
%\bibitem{liu2004} Y. Liu, L. Liu, P. Liu, L. Sheng, S. Fan, Plasma etching carbon nanotube arrays and the field emission properties, Diamond \& Related Materials 13 (2004) 1609-1613
%
%\bibitem{raza2022} M. M. H. Raza, S. M. Aalam, M. Sadiq, M. Sarvar, M. Zulfequar, S. Husain, J. Ali, Time-dependent resonating plasma treatment of carbon nanotubes for enhancing the electron field emission properties, J Mater Sci: Mater Electron (2022) 33:1211-1227
%
%\bibitem{xu2016} M. Xu, F. Du, S. Ganguli, A.Roy, L. Dai, Carbon nanotube dry adhesives with temperature-enhanced adhesion over a large temperature range, Nat Commun 7, 13450 (2016). https://doi.org/10.1038/ncomms13450
%
%\bibitem{seo2021} S. Seo, S. Kim, S. Yamamoto, K. Cui, T. Kodama, J. Shiomi, T. Inoue, S. Chiashi, S. Maruyama, A. J. Hart, Tailoring the surface morpology of carbon nanotube forests by plasma etching: A parametric study, Carbon, 180, 2021, 204-214, https://doi.org/10.1016/j.carbon.2021.04.066
%
%\bibitem{babu2014} D.J. Babu, S. Yadav, T. Heinlein, G. Cherkashinin, J.J. Schneider, Carbon dioxide plasma as a versatile medium for purification and functionalization of vertically aligned carbon nanotubes, J. Phys. Chem. C 118 (2014) 12028e12034, https://doi.org/10.1021/jp5027515
%
%\bibitem{huang2002} S. Huang, L. Dai, Plasma Etching for Purification and Controlled Opening of Aligned Carbon Nanotubes, J. Phys. Chem. B 2002, 106, 3543-3545
%
%\bibitem{hou2008} Z. Hou, B. Cai, H. Liu, D. Xu, Ar, O2, CHF3, and SF6 plasma treatments of screen-printed carbon nanotube films for electrode applications, Carbon, 46, 405-413, 2008
%
%\bibitem{zhao2012} B. Zhao, L. Zhang, X. Wang, J. Yang, Surface functionalization of vertically-aligned carbon nanotube forests by radio-frequency Ar/O$_2$ plasma, Carbon, 50, 2710-2716, 2012
%
%\bibitem{rago2019} Rago I, Rauti R, Bevilacqua M, Calaresu I, Pozzato A, Cibinel M, et al. Carbon Nanotubes, Directly Grown on Supporting Surfaces, Improve Neuronal Activity in Hippocampal Neuronal Networks. Adv Biosyst 2019;3:1800286. doi:10.1002/ADBI.201800286
%
%\bibitem{pampaloni2020} Pampaloni NP, Rago I, Calaresu I, Cozzarini L, Casalis L, Goldoni A, et al. Transparent carbon nanotubes promote the outgrowth of enthorino-dentate projections in lesioned organ slice cultures. Dev Neurobiol 2020;80:316-31. doi:10.1002/DNEU.22711
%


\end{thebibliography}


\end{CJK*}

\end{document}


